%==============================常用宏包、环境==============================%
\documentclass[a4]{ctexart}
\usepackage[a4paper]{geometry}
\RequirePackage{zhnumber}                
\RequirePackage{titlesec, titletoc}
\RequirePackage{tikz,pgf}
\usetikzlibrary{shapes,calc}
\usepackage{xeCJK} % For Chinese characters
\usepackage{amsmath, amsthm}
\usepackage{listings,xcolor}
\usepackage{geometry} % 设置页边距
\usepackage{fontspec}
\usepackage{graphicx}
\usepackage{float} %设置图片浮动位置的宏包
\usepackage{subfigure} %插入多图时用子图显示的宏包
\usepackage{fancyhdr} % 自定义页眉页脚
\setsansfont{Consolas} % 设置英文字体
\setmonofont[Mapping={}]{Consolas} % 英文引号之类的正常显示,相当于设置英文字体
\geometry{left=1cm,right=1cm,top=2cm,bottom=0.5cm} % 页边距
\setlength{\columnsep}{30pt}
% \setlength\columnseprule{0.4pt} % 分割线
%==============================常用宏包、环境==============================%

%==============================页眉、页脚、代码格式设置==============================%
% 页眉、页脚设置
\pagestyle{fancy}
\fancyhf{}
%\lhead[\thepage]{\leftmark} 
% \rhead[\nouppercase{\rightmark}]{\thepage}
 \lhead{CUMTB}
%\lhead[\thepage]{\leftmark} 
\lhead{\CJKfamily{hei} 泡泡猿专用模板}
\rhead{第 \thepage 页}
% \rhead{Page \thepage}
\chead[\thepage]{\leftmark} 
%\lfoot{} 
%\cfoot{}
%\rfoot{}
\renewcommand{\headrulewidth}{0.4pt} 
\renewcommand{\footrulewidth}{0.4pt}

% 代码格式设置
\lstset{
    language    = c++,
    numbers     = left,
    numberstyle = \tiny,
    breaklines  = true,
    captionpos  = b,
    tabsize     = 4,
    frame       = shadowbox,
    columns     = fullflexible,
    commentstyle = \color[RGB]{0,128,0},
    keywordstyle = \color[RGB]{0,0,255},
    basicstyle   = \small\ttfamily,
    stringstyle  = \color[RGB]{148,0,209}\ttfamily,
    rulesepcolor = \color{red!20!green!20!blue!20},
    showstringspaces = false,
}
%==============================页眉、页脚、代码格式设置==============================%

%==============================标题和目录==============================%
\title{\CJKfamily{hei} \bfseries 泡泡猿ACM模板}
\author{Rand0w \& REXWIND \& Dallby}
\renewcommand{\today}{\number\year 年 \number\month 月 \number\day 日}

\begin{document}\small
\begin{titlepage}
\maketitle
\begin{figure}[H] %H为当前位置,!htb为忽略美学标准,htbp为浮动图形
\centering %图片居中
\includegraphics[width=0.7\textwidth]{1.png } %插入图片,[]中设置图片大小,{}中是图片文件名
\end{figure}
\end{titlepage}

\newpage
\pagestyle{empty}
\renewcommand{\contentsname}{目录}
\tableofcontents
\newpage\clearpage
\newpage
\pagestyle{fancy}
\setcounter{page}{1}   %new page
%==============================标题和目录==============================%

%==============================正文部分==============================%
\section{头文件}
\subsection{头文件(Rand0w)}
\begin{lstlisting}
#include <bits/stdc++.h>
//#include <bits/extc++.h>
//using namespace __gnu_pbds;
//using namespace __gnu_cxx;
using namespace std;
#pragma optimize(2)
//#pragma GCC optimize("Ofast,no-stack-protector")
//#pragma GCC target("sse,sse2,sse3,ssse3,sse4,popcnt,abm,mmx,avx,avx2,tune=native")
#define rbset(T) tree<T,null_type,less<T>,rb_tree_tag,tree_order_statistics_node_update>
const int inf = 0x7FFFFFFF;
typedef long long ll;
typedef double db;
typedef long double ld;
template<class T>inline void MAX(T &x,T y){if(y>x)x=y;}
template<class T>inline void MIN(T &x,T y){if(y<x)x=y;}
namespace FastIO
{
char buf[1 << 21], buf2[1 << 21], a[20], *p1 = buf, *p2 = buf, hh = '\n';
int p, p3 = -1;
void read() {}
void print() {}
inline int getc()
{
return p1 == p2 && (p2 = (p1 = buf) + fread(buf, 1, 1 << 21, stdin), p1 == p2) ? EOF : *p1++;
}
inline void flush()
{
fwrite(buf2, 1, p3 + 1, stdout), p3 = -1;
}
template <typename T, typename... T2>
inline void read(T &x, T2 &... oth)
{
int f = 0;x = 0;char ch = getc();
while (!isdigit(ch)){if (ch == '-')f = 1;ch = getc();}
while (isdigit(ch)){x = x * 10 + ch - 48;ch = getc();}
x = f ? -x : x;read(oth...);
}
template <typename T, typename... T2>
inline void print(T x, T2... oth)
{
if (p3 > 1 << 20)flush();
if (x < 0)buf2[++p3] = 45, x = -x;
do{a[++p] = x % 10 + 48;}while (x /= 10);
do{buf2[++p3] = a[p];}while (--p);
buf2[++p3] = hh;
print(oth...);
}
} // namespace FastIO
#define read FastIO::read
#define print FastIO::print
#define flush FastIO::flush
#define spt fixed<<setprecision
#define endll '\n'
#define mul(a,b,mod) (__int128)(a)*(b)%(mod) 
#define pii(a,b) pair<a,b>
#define pow powmod
#define X first
#define Y second
#define lowbit(x) (x&-x)
#define MP make_pair
#define pb push_back
#define pt putchar
#define yx_queue priority_queue
#define lson(pos) (pos<<1)
#define rson(pos) (pos<<1|1)
#define y1 code_by_Rand0w
#define yn A_muban_for_ACM
#define j1 it_is just_an_eastegg
#define lr hope_you_will_be_happy_to_see_this
#define int long long
#define rep(i, a, n) for (register int i = a; i <= n; ++i)
#define per(i, a, n) for (register int i = n; i >= a; --i)
const ll llinf = 4223372036854775851;
const ll mod = (0 ? 1000000007 : 998244353);
ll pow(ll a,ll b,ll md=mod) {ll res=1;a%=md; assert(b>=0); for(;b;b>>=1){if(b&1)res=mul(res,a,md);a=mul(a,a,md);}return res;}
const ll mod2 = 999998639;
const int m1 = 998244353;
const int m2 = 1000001011;
const int pr=233;
const double eps = 1e-7;
const int maxm= 1;
const int maxn = 510000;
void work()
{
	
}
signed main()
{
   #ifndef ONLINE_JUDGE
   //freopen("in.txt","r",stdin);
	//freopen("out.txt","w",stdout);
#endif
	//std::ios::sync_with_stdio(false);
	//cin.tie(NULL);
	int t = 1;
	//cin>>t;
	for(int i=1;i<=t;i++){
		//cout<<"Case #"<<i<<":"<<endll;
		work();
	}
	return 0;
}
\end{lstlisting}
\newpage
\subsection{头文件(REXWind)}
\begin{lstlisting}
#include<iostream>
#include<cstring>
#include<cstdio>
#include<algorithm>
#include<vector>
#include<map>
#include<queue>
#include<cmath>
using namespace std;

template<class T>inline void read(T &x){
	x=0;char o,f=1;
	while(o=getchar(),o<48)if(o==45)f=-f;
	do x=(x<<3)+(x<<1)+(o^48);while(o=getchar(),o>47);x*=f;}
int cansel_sync=(ios::sync_with_stdio(0),cin.tie(0),0);
#define ll long long
#define ull unsigned long long
#define rep(i,a,b) for(int i=(a);i<=(b);i++)
#define repb(i,a,b) for(int i=(a);i>=b;i--)
#define mkp make_pair
#define ft first
#define sd second
#define log(x) (31-__builtin_clz(x))
#define INF 0x3f3f3f3f
typedef pair<int,int> pii;
typedef pair<ll,ll> pll;
ll gcd(ll a,ll b){ while(b^=a^=b^=a%=b); return a; }
//#define INF 0x7fffffff

void solve(){
	
}

int main(){
	int z;
	cin>>z;
	while(z--) solve();
}
\end{lstlisting}
\subsection{头文件(Dallby)}
\begin{lstlisting}
#include<bits/stdc++.h>
cout<<"hello<<endl;
\end{lstlisting}

\section{数论}

\subsection{欧拉筛}
$O(n)$筛素数
\begin{lstlisting}
int primes[maxn+5],tail;
bool is_prime[maxn+5];
void euler(){
   is_prime[1] = 1;
   for (int i = 2; i < maxn; i++)
   {
      if (!is_prime[i])
      primes[++tail]=i;
      for (int j = 0; j < primes.size() && i * primes[j] < maxn; j++)
      {
         is_prime[i * primes[j]] = 1;
         if ((i % primes[j]) == 0)
            break;
      }
   }
}
\end{lstlisting}

\subsection{Exgcd}
求出$ax+by=gcd(a,b)$的一组可行解 $O(logn)$ 
\begin{lstlisting}
void Exgcd(ll a,ll b,ll &d,ll &x,ll &y){
	if(!b){d=a;x=1;y=0;}
	else{Exgcd(b,a%b,d,y,x);y-=x*(a/b);}
}
\end{lstlisting}

\subsection{Excrt 扩展中国剩余定理}
求解同余方程组
$\begin{cases}
	\begin{aligned}
	x \ \% \ b_1  &\equiv \  a_1\\
	x \ \% \ b_2  &\equiv \ a_2\\
	           		& \ \vdots   \\
	x \ \% \ b_n  &\equiv  \ a_n
	\end{aligned}
\end{cases}$
\begin{lstlisting}
int excrt(int a[],int b[],int n){
    int lc=1;
    for(int i=1;i<=n;i++)
        lc=lcm(lc,a[i]);
    for(int i=1;i<n;i++){
        int p,q,g;
        g=exgcd(a[i],a[i+1],p,q);
        int k=(b[i+1]-b[i])/g;
        q=-q;p*=k;q*=k;
        b[i+1]=a[i]*p%lc+b[i];
        b[i+1]%=lc;
        a[i+1]=lcm(a[i],a[i+1]);
    }
    return (b[n]%lc+lc)%lc;
}
\end{lstlisting}

\subsection{线性求逆元}
\begin{lstlisting}
void init(int p){
	inv[1] = 1;
	for(int i=2;i<=n;i++){
		inv[i] = (ll)(p-p/i)*inv[p%i]%p;
	}
}
\end{lstlisting}
\subsection{多项式}
\subsubsection{FFT快速傅里叶变换}
\begin{lstlisting}
const int SIZE=(1<<21)+5;
const double PI=acos(-1);
struct CP{
    double x,y;
    CP(double x=0,double y=0):x(x),y(y){}
    CP operator +(const CP &A)const{return CP(x+A.x,y+A.y);}
    CP operator -(const CP &A)const{return CP(x-A.x,y-A.y);}
    CP operator *(const CP &A)const{return CP(x*A.x-y*A.y,x*A.y+y*A.x);}
};
int limit,rev[SIZE];
void DFT(CP *F,int op){
    for(int i=0;i<limit;i++)if(i<rev[i])swap(F[i],F[rev[i]]);
    for(int mid=1;mid<limit;mid<<=1){
        CP wn(cos(PI/mid),op*sin(PI/mid));
        for(int len=mid<<1,k=0;k<limit;k+=len){
            CP w(1,0);
            for(int i=k;i<k+mid;i++){
                CP tmp=w*F[i+mid];
                F[i+mid]=F[i]-tmp;
                F[i]=F[i]+tmp;
                w=w*wn;
            }
        }
    }
    if(op==-1)for(int i=0;i<limit;i++)F[i].x/=limit;
}
void FFT(int n,int m,CP *F,CP *G){
    for(limit=1;limit<=n+m;limit<<=1);
    for(int i=0;i<limit;i++)rev[i]=(rev[i>>1]>>1)|((i&1)?limit>>1:0);
    DFT(F,1),DFT(G,1);
    for(int i=0;i<limit;i++)F[i]=F[i]*G[i];
    DFT(F,-1);
}
\end{lstlisting}
\subsubsection{NTT快速数论变换}
\begin{lstlisting}
const int SIZE=(1<<21)+5;
int limit,rev[SIZE];
void DFT(ll *f, int op) {
    const ll G = 3;
    for(int i=0; i<limit; ++i) if(i<rev[i]) swap(f[i],f[rev[i]]);
    for(int len=2; len<=limit; len<<=1) {
        ll w1=pow(pow(G,(mod-1)/len),~op?1:mod-2);
        for(int l=0, hf=len>>1; l<limit; l+=len) {
            ll w=1;
            for(int i=l; i<l+hf; ++i) {
                ll tp=w*f[i+hf]%mod;
                f[i+hf]=(f[i]-tp+mod)%mod;
                f[i]=(f[i]+tp)%mod;
                w=w*w1%mod;
            }
        }
    }
    if(op==-1) for(int i=0, inv=pow(limit,mod-2); i<limit; ++i) f[i]=f[i]*inv%mod;
}
void NTT(int n,int m,int *F,int *G){
    for(limit=1;limit<=n+m;limit<<=1);
    for(int i=0;i<limit;i++)rev[i]=(rev[i>>1]>>1)|((i&1)?limit>>1:0);
    DFT(F,1),DFT(G,1);
    for(int i=0;i<limit;i++)F[i]=F[i]*G[i];
    DFT(F,-1);
}
\end{lstlisting}
\subsubsection{MTT任意模数FFT}
FFT版常数巨大,慎用。
\begin{lstlisting}
struct MTT{
    static const int N=1<<20;
    struct cp{
        long double a,b;
        cp(){a=0,b=0;}
        cp(const long double &a,const long double &b):a(a),b(b){}
        cp operator+(const cp &t)const{return cp(a+t.a,b+t.b);}
        cp operator-(const cp &t)const{return cp(a-t.a,b-t.b);}
        cp operator*(const cp &t)const{return cp(a*t.a-b*t.b,a*t.b+b*t.a);}
        cp conj()const{return cp(a,-b);}
    };
    cp wn(int n,int f){
        static const long double pi=acos(-1.0);
        return cp(cos(pi/n),f*sin(pi/n));
    }
    int g[N];
    void dft(cp a[],int n,int f){
        for(int i=0;i<n;i++)if(i>g[i])swap(a[i],a[g[i]]);
        for(int i=1;i<n;i<<=1){
            cp w=wn(i,f);
            for(int j=0;j<n;j+=i<<1){
                cp e(1,0);
                for(int k=0;k<i;e=e*w,k++){
                    cp x=a[j+k],y=a[j+k+i]*e;
                    a[j+k]=x+y,a[j+k+i]=x-y;
                }
            }
        }
        if(f==-1){
            cp Inv(1.0/n,0);
            for(int i=0;i<n;i++)a[i]=a[i]*Inv;
        }
    }
    cp a[N],b[N],Aa[N],Ab[N],Ba[N],Bb[N];
    vector<ll> conv_mod(const vector<ll> &u,const vector<ll> &v,ll mod){ // 任意模数fft
        const int n=(int)u.size()-1,m=(int)v.size()-1,M=sqrt(mod)+1;
        const int k=32-__builtin_clz(n+m+1),s=1<<k;
        g[0]=0; for(int i=1;i<s;i++)g[i]=(g[i/2]/2)|((i&1)<<(k-1));
        for(int i=0;i<s;i++){
            a[i]=i<=n?cp(u[i]%mod%M,u[i]%mod/M):cp();
            b[i]=i<=m?cp(v[i]%mod%M,v[i]%mod/M):cp();
        }
        dft(a,s,1); dft(b,s,1);
        for(int i=0;i<s;i++){
            int j=(s-i)%s;
            cp t1=(a[i]+a[j].conj())*cp(0.5,0);
            cp t2=(a[i]-a[j].conj())*cp(0,-0.5);
            cp t3=(b[i]+b[j].conj())*cp(0.5,0);
            cp t4=(b[i]-b[j].conj())*cp(0,-0.5);
            Aa[i]=t1*t3,Ab[i]=t1*t4,Ba[i]=t2*t3,Bb[i]=t2*t4;
        }
        for(int i=0;i<s;i++){
            a[i]=Aa[i]+Ab[i]*cp(0,1);
            b[i]=Ba[i]+Bb[i]*cp(0,1);
        }
        dft(a,s,-1); dft(b,s,-1);
        vector<ll> ans;
        for(int i=0;i<n+m+1;i++){
            ll t1=llround(a[i].a)%mod;
            ll t2=llround(a[i].b)%mod;
            ll t3=llround(b[i].a)%mod;
            ll t4=llround(b[i].b)%mod;
            ans.push_back((t1+(t2+t3)*M%mod+t4*M*M)%mod);
        }
        return ans;
    }
}mtt;
\end{lstlisting}
\subsection{组合数}
预处理阶乘,并通过逆元实现相除
\begin{lstlisting}
ll jc[MAXN];
ll qpow(ll d,ll c){//快速幂
    ll res = 1;
    while(c){
        if(c&1) res=res*d%med;
        d=d*d%med;c>>=1;
    }return res;
}
inline ll niyuan(ll x){return qpow(x,med-2);}
void initjc(){//初始化阶乘
    jc[0] = 1;
    rep(i,1,MAXN-1) jc[i] = jc[i-1]*i%med;
}
inline int C(int n,int m){//n是下面的
    if(n<m) return 0;
    return jc[n]*niyuan(jc[n-m])%med*niyuan(jc[m])%med;
}
int main(){
    initjc();
    int n,m;
    while(cin>>n>>m) cout<<C(n,m)<<endl;
}
\end{lstlisting}

\subsection{矩阵快速幂}
\begin{lstlisting}
struct Matrix{
	ll a[MAXN][MAXN];
	Matrix(ll x=0){
		for(int i=0;i<n;i++){
			for(int j=0;j<n;j++){
				a[i][j]=x*(i==j);
			}
		}
	}
	Matrix operator *(const Matrix &b)const{//通过重载运算符实现矩阵乘法 
		Matrix res(0);
		for(int i=0;i<n;i++){
			for(int j=0;j<n;j++){
				for(int k=0;k<n;k++){
					ll &ma = res.a[i][j];
					ma = (ma+a[i][k]*b.a[k][j])%mod;
				}
			}
		}
		return res;
	}
};
Matrix qpow(Matrix d,ll m){//底数和幂次数 
	Matrix res(1);//构造E单位矩阵 
	while(m){
		if(m&1) 
			res=res*d;
		d=d*d;
		m>>=1;
	}
	return res; 
}
\end{lstlisting}

\subsection{高斯消元}
$O(n^3)$复杂度,需要用double存储。
\begin{lstlisting}
double date[110][110];
bool guass(int n){
    for(int i=1;i<=n;i++){
        int mix=-1;
        for(int j=i;j<=n;j++)
            if(date[j][i]!=0){
                mix=j;break;
            }
        if(mix==-1)
            return false;
        if(mix!=i)
            for(int j=1;j<=n+1;j++)
                swap(date[mix][j],date[i][j]);
        double t=date[i][i];
        for(int j=i;j<=n+1;j++){
            date[i][j]=date[i][j]/t;
        }
        for(int j=1;j<=n;j++){
            if(date[j][i]==0||j==i)
                continue;
            double g=date[j][i]/date[i][i];
            for(int k=1;k<=n+1;k++)
                date[j][k]-=date[i][k]*g;
        }
    }                                                                         
    return true;
}
\end{lstlisting}

\subsection{三点求圆心}
\begin{lstlisting}
struct point{
	double x;
	double y;
};

point cal(point a,point b,point c){
	double x1 = a.x;double y1 = a.y;
	double x2 = b.x;double y2 = b.y;
	double x3 = c.x; double y3 = c.y;
	double a1 = 2*(x2-x1); double a2 = 2*(x3-x2);
	double b1 = 2*(y2-y1); double b2 = 2*(y3-y2);
	double c1 = x2*x2 + y2*y2 - x1*x1 - y1*y1;
	double c2 = x3*x3 + y3*y3 - x2*x2 - y2*y2;
	double rx = (c1*b2-c2*b1)/(a1*b2-a2*b1);
	double ry = (c2*a1-c1*a2)/(a1*b2-a2*b1);
	return point{rx,ry};
}
\end{lstlisting}

\subsection{欧拉降幂}
$$
a^b \equiv \begin{cases}
a^{b\%\phi(p)} , & \gcd(a,p)=1\\
a^b , & \gcd(a,p)\neq 1,b<\phi(p)\\
a^{b\%\phi(p)+\phi(p)} , & \gcd(a,p)\neq 1 , b\geq \phi(p)\\
\end{cases}
(\mod p)
$$
\subsection{拉格朗日插值}
\begin{lstlisting}
namespace polysum {
#define rep(i,a,n) for (int i=a;i<n;i++)
#define per(i,a,n) for (int i=n-1;i>=a;i--)
const int D = 1010000; ///可能需要用到的最高次
LL a[D], f[D], g[D], p[D], p1[D], p2[D], b[D], h[D][2], C[D];
LL powmod(LL a, LL b) {
    LL res = 1;
    a %= mod;
    assert(b >= 0);

    for (; b; b >>= 1) {
        if (b & 1)
            res = res * a % mod;

        a = a * a % mod;
    }

    return res;
}

///函数用途:给出数列的(d+1)项,其中d为最高次方项
///求出数列的第n项,数组下标从0开始
LL calcn(int d, LL *a, LL n) { /// a[0].. a[d]  a[n]
    if (n <= d)
        return a[n];

    p1[0] = p2[0] = 1;
    rep(i, 0, d + 1) {
        LL t = (n - i + mod) % mod;
        p1[i + 1] = p1[i] * t % mod;
    }
    rep(i, 0, d + 1) {
        LL t = (n - d + i + mod) % mod;
        p2[i + 1] = p2[i] * t % mod;
    }
    LL ans = 0;
    rep(i, 0, d + 1) {
        LL t = g[i] * g[d - i] % mod * p1[i] % mod * p2[d - i] % mod * a[i] % mod;

        if ((d - i) & 1)
            ans = (ans - t + mod) % mod;
        else
            ans = (ans + t) % mod;
    }
    return ans;
}
void init(int M) {///用到的最高次
    f[0] = f[1] = g[0] = g[1] = 1;
    rep(i, 2, M + 5) f[i] = f[i - 1] * i % mod;
    g[M + 4] = powmod(f[M + 4], mod - 2);
    per(i, 1, M + 4) g[i] = g[i + 1] * (i + 1) % mod; ///费马小定理筛逆元
}

///函数用途:给出数列的(m+1)项,其中m为最高次方
///求出数列的前(n-1)项的和(从第0项开始)
LL polysum(LL m, LL *a, LL n) { /// a[0].. a[m] \sum_{i=0}^{n-1} a[i]
    for (int i = 0; i <= m; i++)
        b[i] = a[i];

    ///前n项和,其最高次幂加1
    b[m + 1] = calcn(m, b, m + 1);
    rep(i, 1, m + 2) b[i] = (b[i - 1] + b[i]) % mod;
    return calcn(m + 1, b, n - 1);
}
LL qpolysum(LL R, LL n, LL *a, LL m) { /// a[0].. a[m] \sum_{i=0}^{n-1} a[i]*R^i
    if (R == 1)
        return polysum(n, a, m);

    a[m + 1] = calcn(m, a, m + 1);
    LL r = powmod(R, mod - 2), p3 = 0, p4 = 0, c, ans;
    h[0][0] = 0;
    h[0][1] = 1;
    rep(i, 1, m + 2) {
        h[i][0] = (h[i - 1][0] + a[i - 1]) * r % mod;
        h[i][1] = h[i - 1][1] * r % mod;
    }
    rep(i, 0, m + 2) {
        LL t = g[i] * g[m + 1 - i] % mod;

        if (i & 1)
            p3 = ((p3 - h[i][0] * t) % mod + mod) % mod, p4 = ((p4 - h[i][1] * t) % mod + mod) % mod;
        else
            p3 = (p3 + h[i][0] * t) % mod, p4 = (p4 + h[i][1] * t) % mod;
    }
    c = powmod(p4, mod - 2) * (mod - p3) % mod;
    rep(i, 0, m + 2) h[i][0] = (h[i][0] + h[i][1] * c) % mod;
    rep(i, 0, m + 2) C[i] = h[i][0];
    ans = (calcn(m, C, n) * powmod(R, n) - c) % mod;

    if (ans < 0)
        ans += mod;

    return ans;
}
}
\end{lstlisting}
\section{数据结构}
\subsection{并查集系列}
\subsubsection{普通并查集}
带路径压缩,$O(1)$复杂度
\begin{lstlisting}
int fa[maxn];
int find(int x){if(fa[x]^x)return fa[x]=find(fa[x]);return x;}
void merge(int a,int b){fa[find(a)]=find(b);}	
\end{lstlisting}
\subsubsection{按秩合并并查集}
\begin{lstlisting}
int fa[maxn];
int dep[maxn];
int find(int x){int now=x; while(fa[now]^now)now=fa[now];return now;}
void merge(int a,int b){
    int l=find(a),r=find(b);
    if(l==r) return;
    if(dep[l]>dep[r])swap(l,r);
    fa[l]=r;
    dep[r]+=dep[l]==dep[r];
}
\end{lstlisting}
\subsubsection{可持久化并查集}
\begin{lstlisting}
struct chair_man_tree{
    struct node{
        int lson,rson;
    }tree[maxn<<5];
    int tail=0;
    int tail2=0;
    int fa[maxn<<2];
    int depth[maxn<<2];
    inline int getnew(int pos){
        tree[++tail]=tree[pos];
        return tail;
    }
    int build(int l,int r){
        
        if(l==r){
            fa[++tail2]=l;
            depth[tail2]=1;
            return tail2;
        }
        int now=tail++;
        int mid=(l+r)>>1;
        tree[now].lson=build(l,mid);
        tree[now].rson=build(mid+1,r);
        return now;
    }
    int query(int pos,int l,int r,int qr){
        if(l==r)
            return pos;
        int mid=(l+r)>>1;
        if(qr<=mid)
            return query(tree[pos].lson,l,mid,qr);
        else return query(tree[pos].rson,mid+1,r,qr);
    }
    int update(int pos,int l,int r,int qr,int val){
        if(l==r){
            depth[++tail2]=depth[pos];
            fa[tail2]=val;
            return tail2;
        }
        int now=getnew(pos);
        int mid=(l+r)>>1;
        if(mid>=qr)
            tree[now].lson=update(tree[now].lson,l,mid,qr,val);
        else tree[now].rson=update(tree[now].rson,mid+1,r,qr,val);
        return now;
    }
    int add(int pos,int l,int r,int qr){
        if(l==r){
            depth[++tail2]=depth[pos]+1;
            fa[tail2]=fa[pos];
            return tail2;
        }
        int now=getnew(pos);
        int mid=(l+r)>>1;
        if(mid>=qr)
            tree[now].lson=add(tree[now].lson,l,mid,qr);
        else tree[now].rson=add(tree[now].rson,mid+1,r,qr);
        return now;
    }
    int getfa(int root,int qr){
        int t=fa[query(root,1,n,qr)];
        if(qr==t)
        return qr;
        else return getfa(root,t);
    }
}t;
\end{lstlisting}
\subsubsection{ETT维护动态图连通性}
待补
\subsection{平衡树系列}
\subsubsection{fhq\_treap}
无旋treap,可持久化,常数大
\begin{lstlisting}
mt19937 rnd(514114);
struct fhq_treap{
    struct node{
        int l, r;
        int val, key;
        int size;
    } fhq[maxn];
    int cnt, root;
    inline int newnode(int val){
        fhq[++cnt].val = val;
        fhq[cnt].key = rnd();
        fhq[cnt].size = 1;
        fhq[cnt].l = fhq[cnt].r = 0;
        return cnt;
    }
    inline void pushup(int now){
    fhq[now].size = fhq[fhq[now].l].size + fhq[fhq[now].r].size + 1;
    }
    void split(int now, int val, int &x, int &y){
        if (!now){
            x = y = 0;
            return;
        }
        else if (fhq[now].val <= val){
        x = now;
        split(fhq[now].r, val, fhq[now].r, y);
        }
        else{
        y = now;
        split(fhq[now].l, val, x, fhq[now].l);
        }
    pushup(now);
    }
    int merge(int x, int y){
        if (!x || !y)
            return x + y;
        if (fhq[x].key > fhq[y].key){
            fhq[x].r = merge(fhq[x].r, y);
            pushup(x);
            return x;
        }else{
            fhq[y].l = merge(x, fhq[y].l);
            pushup(y);
            return y;
        }
    }
    inline void insert(int val){
        int x, y;
        split(root, val, x, y);
        root = merge(merge(x, newnode(val)), y);
    }
    inline void del(int val){
        int x, y, z;
        split(root, val - 1, x, y);
        split(y, val, y, z);
        y = merge(fhq[y].l, fhq[y].r);
        root = merge(merge(x, y), z);
    }
    inline int getrk(int num){
        int x, y;
        split(root, num - 1, x, y);
        int ans = fhq[x].size + 1;
        root = merge(x, y);
        return ans;
    }
    inline int getnum(int rank){
        int now=root;
        while(now)
        {
            if(fhq[fhq[now].l].size+1==rank)
               break;
            else if(fhq[fhq[now].l].size>=rank)
                now=fhq[now].l;
            else{
                rank-=fhq[fhq[now].l].size+1;
                now=fhq[now].r;
            }
        }
        return fhq[now].val;
    }
    inline int pre(int val){
        int x, y, ans;
        split(root, val - 1, x, y);
        int t = x;
        while (fhq[t].r)
            t = fhq[t].r;
        ans = fhq[t].val;
        root = merge(x, y);
        return ans;
    }
    inline int aft(int val){
        int x, y, ans;
        split(root, val, x, y);
        int t = y;
        while (fhq[t].l)
            t = fhq[t].l;
        ans = fhq[t].val;
        root = merge(x, y);
        return ans;
    }
} tree;
\end{lstlisting}
\section{字符串}
\subsection{FFT解决字符串匹配问题}
可以用来解决含有通配符的字符串匹配问题
定义匹配函数 $$(x,y) = (A_x-B_x)^2$$
如果两个字符相同,则满足 $C(x,y)=0$\\
定义模式串和文本串x位置对齐时候的完全匹配函数为
$$P(x)=\sum C(i,x+i)$$
模式串在位置x上匹配时,$p(x)=0$\\
通过将模式串reverse后卷积,可以快速处理每个位置x上的完全匹配函数$P(x)$
同理,如果包含通配符,则设通配符的值为0,可以构造损失函数
$$C(x,y)=(A_x-B_x)^2 \cdot A_x \cdot B_x=A_x^3 B_x+A_xB_x^3-2A_x^2B_x^2$$
通过三次FFT即可求得每个位置上的P(x)

\subsection{后缀数组SA+LCP}
LCP(i,j) 后缀i和后缀j的最长公共前缀
\begin{lstlisting}
int n,m;
string s;
int rk[MAXN],sa[MAXN],c[MAXN],rk2[MAXN];
//sa[i]存排名i的原始编号 rk[i]存编号i的排名 第二关键字rk2
inline void get_SA(){
    rep(i,1,n) ++c[rk[i]=s[i]];//基数排序
    rep(i,2,m) c[i] += c[i-1];
    //c做前缀和,可以知道每个关键字的排名最低在哪里
    repb(i,n,1) sa[c[rk[i]]--] = i;//记录每个排名的原编号

    for(int w=1;w<=n;w<<=1){//倍增
        int num = 0;
        rep(i,n-w+1,n) rk2[++num] = i;//没有第二关键字的排在前面
        rep(i,1,n) if(sa[i]>w) rk2[++num] = sa[i]-w;
        //编号sa[i]大于w的才能作为编号sa[i]-w的第二关键字
        rep(i,1,m) c[i] = 0;
        rep(i,1,n) ++c[rk[i]];
        rep(i,2,m) c[i]+=c[i-1];
        repb(i,n,1) sa[c[rk[rk2[i]]]--]=rk2[i],rk2[i]=0;
        //同一个桶中按照第二关键字排序
        swap(rk,rk2);
        //这时候的rk2时这次排序用到的上一轮的rk,要计算出新的rk给下一轮排序

        rk[sa[1]]=1,num=1;
        rep(i,2,n)
            rk[sa[i]] = (rk2[sa[i]]==rk2[sa[i-1]]&&rk2[sa[i]+w]==rk2[sa[i-1]+w])?num:++num;
        //下一次排名的第一关键字,相同的两个元素排名也相同
        if(num==n) break;//rk都唯一时,排序结束
        m=num;
    }
}
int height[MAXN];
inline void get_height(){
    int k = 0,j;
    rep(i,1,n) rk[sa[i]] = i;
    rep(i,1,n){
        if(rk[i]==1) continue;//第一名往前没有前缀
        if(k) k--;//h[i]>=h[i-1]-1 即height[rk[i]]>=height[rk[i-1]]-1
        j = sa[rk[i]-1];//找排在rk[i]前面的
        while(j+k<=n&&i+k<=n&&s[i+k]==s[j+k]) ++k;//逐字符比较
        //因为每次k只会-1,故++k最多只会加2n次
        height[rk[i]] = k;
    }
}
inline void solve(){
    cin>>s;
    s = ' '+s;
    n = s.size()-1,m = 122;//m为字符个数'z'=122
    get_SA();
    rep(i,1,n) cout<<sa[i]<<' ';
    cout<<endl;
}
\end{lstlisting}

\section{杂项}

\subsection{集合set}
还可以通过lower\_bound和upper\_bound返回迭代器来找前驱,后继
\begin{lstlisting}
//并交集
vector<int> ANS;
set_union(s1.begin(),s1.end(),s2.begin(),s2.end(),inserter(ANS,ANS.begin()));//set_intersection()

//通过迭代器遍历集合
set<char>::iterator iter = temp1.begin();
while (iter!=temp1.end()){
	cout<<*iter;
	iter++;
}
\end{lstlisting}

\subsection{快读快写(短)}
\begin{lstlisting}
template<class T>inline void read(T &x){x=0;char o,f=1;while(o=getchar(),o<48)if(o==45)f=-f;do x=(x<<3)+(x<<1)+(o^48);while(o=getchar(),o>47);x*=f;}
template<class T>
void wt(T x){//快写
   if(x < 0) putchar('-'), x = -x;
   if(x >= 10) wt(x / 10);
   putchar('0' + x % 10);
}
\end{lstlisting}

\subsection{GCD(压行)}
\begin{lstlisting}
ll gcd(ll a,ll b){ while(b^=a^=b^=a%=b); return a; }
\end{lstlisting}

\subsection{计时}
\begin{lstlisting}
inline double run_time(){
    return 1.0*clock()/CLOCKS_PER_SEC;
}
\end{lstlisting}
%==============================正文部分==============================%
\end{document}