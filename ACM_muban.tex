%==============================常用宏包、环境==============================%
\documentclass[twocolumn,a4]{article}
\usepackage{xeCJK} % For Chinese characters
\usepackage{amsmath, amsthm}
\usepackage{listings,xcolor}
\usepackage{geometry} % 设置页边距
\usepackage{fontspec}
\usepackage{graphicx}
\usepackage{float} %设置图片浮动位置的宏包
\usepackage{subfigure} %插入多图时用子图显示的宏包
\usepackage{fancyhdr} % 自定义页眉页脚
\setsansfont{Consolas} % 设置英文字体
\setmonofont[Mapping={}]{Consolas} % 英文引号之类的正常显示,相当于设置英文字体
\geometry{left=1cm,right=1cm,top=2cm,bottom=0.5cm} % 页边距
\setlength{\columnsep}{30pt}
% \setlength\columnseprule{0.4pt} % 分割线
%==============================常用宏包、环境==============================%

%==============================页眉、页脚、代码格式设置==============================%
% 页眉、页脚设置
\pagestyle{fancy}
% \lhead{CUMTB}
\lhead{\CJKfamily{hei} 泡泡猿专用模板}
\rhead{第 \thepage 页}
% \rhead{Page \thepage}
\chead{\leftmark} 
\lfoot{} 
\cfoot{}
\rfoot{}
\renewcommand{\headrulewidth}{0.4pt} 
\renewcommand{\footrulewidth}{0.4pt}

% 代码格式设置
\lstset{
    language    = c++,
    numbers     = left,
    numberstyle = \tiny,
    breaklines  = true,
    captionpos  = b,
    tabsize     = 4,
    frame       = shadowbox,
    columns     = fullflexible,
    commentstyle = \color[RGB]{0,128,0},
    keywordstyle = \color[RGB]{0,0,255},
    basicstyle   = \small\ttfamily,
    stringstyle  = \color[RGB]{148,0,209}\ttfamily,
    rulesepcolor = \color{red!20!green!20!blue!20},
    showstringspaces = false,
}
%==============================页眉、页脚、代码格式设置==============================%

%==============================标题和目录==============================%
\title{\CJKfamily{hei} \bfseries 泡泡猿ACM模板}
\author{Rand0w \& REXWIND \& Dallby}
\renewcommand{\today}{\number\year 年 \number\month 月 \number\day 日}

\begin{document}\small
\begin{titlepage}
\maketitle
\begin{figure}[H] %H为当前位置,!htb为忽略美学标准,htbp为浮动图形
\centering %图片居中
\includegraphics[width=0.7\textwidth]{1.jpg} %插入图片,[]中设置图片大小,{}中是图片文件名
\end{figure}
\end{titlepage}

\newpage
\pagestyle{empty}
\renewcommand{\contentsname}{目录}
\tableofcontents
\newpage\clearpage
\newpage
\pagestyle{fancy}
\setcounter{page}{1}   %new page
%==============================标题和目录==============================%

%==============================正文部分==============================%
\section{头文件}
\subsection{头文件(Rand0w)}
\begin{lstlisting}
#include <bits/stdc++.h>
//#include <bits/extc++.h>
//using namespace __gnu_pbds;
//using namespace __gnu_cxx;
using namespace std;
#pragma optimize(2)
//#pragma GCC optimize("Ofast,no-stack-protector")
//#pragma GCC target("sse,sse2,sse3,ssse3,sse4,popcnt,abm,mmx,avx,avx2,tune=native")
#define rbset(T) tree<T,null_type,less<T>,rb_tree_tag,tree_order_statistics_node_update>
const int inf = 0x7FFFFFFF;
typedef long long ll;
typedef double db;
typedef long double ld;
template<class T>inline void MAX(T &x,T y){if(y>x)x=y;}
template<class T>inline void MIN(T &x,T y){if(y<x)x=y;}
namespace FastIO
{
char buf[1 << 21], buf2[1 << 21], a[20], *p1 = buf, *p2 = buf, hh = '\n';
int p, p3 = -1;
void read() {}
void print() {}
inline int getc()
{
return p1 == p2 && (p2 = (p1 = buf) + fread(buf, 1, 1 << 21, stdin), p1 == p2) ? EOF : *p1++;
}
inline void flush()
{
fwrite(buf2, 1, p3 + 1, stdout), p3 = -1;
}
template <typename T, typename... T2>
inline void read(T &x, T2 &... oth)
{
int f = 0;x = 0;char ch = getc();
while (!isdigit(ch)){if (ch == '-')f = 1;ch = getc();}
while (isdigit(ch)){x = x * 10 + ch - 48;ch = getc();}
x = f ? -x : x;read(oth...);
}
template <typename T, typename... T2>
inline void print(T x, T2... oth)
{
if (p3 > 1 << 20)flush();
if (x < 0)buf2[++p3] = 45, x = -x;
do{a[++p] = x % 10 + 48;}while (x /= 10);
do{buf2[++p3] = a[p];}while (--p);
buf2[++p3] = hh;
print(oth...);
}
} // namespace FastIO
#define read FastIO::read
#define print FastIO::print
#define flush FastIO::flush
#define spt fixed<<setprecision
#define endll '\n'
#define mul(a,b,mod) (__int128)(a)*(b)%(mod) 
#define pii(a,b) pair<a,b>
#define pow powmod
#define X first
#define Y second
#define lowbit(x) (x&-x)
#define MP make_pair
#define pb push_back
#define pt putchar
#define yx_queue priority_queue
#define lson(pos) (pos<<1)
#define rson(pos) (pos<<1|1)
#define y1 code_by_Rand0w
#define yn A_muban_for_ACM
#define j1 it_is just_an_eastegg
#define lr hope_you_will_be_happy_to_see_this
#define int long long
#define rep(i, a, n) for (register int i = a; i <= n; ++i)
#define per(i, a, n) for (register int i = n; i >= a; --i)
const ll llinf = 4223372036854775851;
const ll mod = (0 ? 1000000007 : 998244353);
ll pow(ll a,ll b,ll md=mod) {ll res=1;a%=md; assert(b>=0); for(;b;b>>=1){if(b&1)res=mul(res,a,md);a=mul(a,a,md);}return res;}
const ll mod2 = 999998639;
const int m1 = 998244353;
const int m2 = 1000001011;
const int pr=233;
const double eps = 1e-7;
const int maxm= 1;
const int maxn = 510000;
void work()
{
	
}
signed main()
{
   #ifndef ONLINE_JUDGE
   //freopen("in.txt","r",stdin);
	//freopen("out.txt","w",stdout);
#endif
	//std::ios::sync_with_stdio(false);
	//cin.tie(NULL);
	int t = 1;
	//cin>>t;
	for(int i=1;i<=t;i++){
		//cout<<"Case #"<<i<<":"<<endll;
		work();
	}
	return 0;
}
\end{lstlisting}
\newpage
\subsection{头文件(REXWind)}
\begin{lstlisting}
#include<iostream>
#include<cstring>
#include<cstdio>
#include<algorithm>
#include<vector>
#include<map>
#include<queue>
#include<cmath>
using namespace std;

template<class T>inline void read(T &x){x=0;char o,f=1;while(o=getchar(),o<48)if(o==45)f=-f;do x=(x<<3)+(x<<1)+(o^48);while(o=getchar(),o>47);x*=f;}
int cansel_sync=(ios::sync_with_stdio(0),cin.tie(0),0);
#define ll long long
#define ull unsigned long long
#define rep(i,a,b) for(int i=(a);i<=(b);i++)
#define repb(i,a,b) for(int i=(a);i>=b;i--)
#define mkp make_pair
#define ft first
#define sd second
#define log(x) (31-__builtin_clz(x))
#define INF 0x3f3f3f3f
typedef pair<int,int> pii;
typedef pair<ll,ll> pll;
ll gcd(ll a,ll b){ while(b^=a^=b^=a%=b); return a; }
//#define INF 0x7fffffff

void solve(){
	
}

int main(){
	int z;
	cin>>z;
	while(z--) solve();
}
\end{lstlisting}
\subsection{头文件(Dallby)}
\begin{lstlisting}
#include<bits/stdc++.h>
cout<<"hello<<endl;
\end{lstlisting}

\section{数论}

\subsection{欧拉筛}
$O(n)$筛素数
\begin{lstlisting}
int primes[maxn+5],tail;
bool is_prime[maxn+5];
void euler(){
   is_prime[1] = 1;
   for (int i = 2; i < maxn; i++)
   {
      if (!is_prime[i])
      primes[++tail]=i;
      for (int j = 0; j < primes.size() && i * primes[j] < maxn; j++)
      {
         is_prime[i * primes[j]] = 1;
         if ((i % primes[j]) == 0)
            break;
      }
   }
}
\end{lstlisting}

\subsection{Exgcd}
求出$ax+by=gcd(a,b)$的一组可行解 $O(logn)$ 
\begin{lstlisting}
void Exgcd(ll a,ll b,ll &d,ll &x,ll &y){
	if(!b){d=a;x=1;y=0;}
	else{Exgcd(b,a%b,d,y,x);y-=x*(a/b);}
}
\end{lstlisting}

\subsection{Excrt 扩展中国剩余定理}
求解同余方程组
$\begin{cases}
	\begin{aligned}
	x \ \% \ b_1  &\equiv \  a_1\\
	x \ \% \ b_2  &\equiv \ a_2\\
	           		& \ \vdots   \\
	x \ \% \ b_n  &\equiv  \ a_n
	\end{aligned}
\end{cases}$
\begin{lstlisting}
int excrt(int a[],int b[],int n){
    int lc=1;
    for(int i=1;i<=n;i++)
        lc=lcm(lc,a[i]);
    for(int i=1;i<n;i++){
        int p,q,g;
        g=exgcd(a[i],a[i+1],p,q);
        int k=(b[i+1]-b[i])/g;
        q=-q;p*=k;q*=k;
        b[i+1]=a[i]*p%lc+b[i];
        b[i+1]%=lc;
        a[i+1]=lcm(a[i],a[i+1]);
    }
    return (b[n]%lc+lc)%lc;
}
\end{lstlisting}

\subsection{线性求逆元}
\begin{lstlisting}
void init(int p){
	inv[1] = 1;
	for(int i=2;i<=n;i++){
		inv[i] = (ll)(p-p/i)*inv[p%i]%p;
	}
}
\end{lstlisting}

\subsection{组合数}
预处理阶乘,并通过逆元实现相除
\begin{lstlisting}
ll jc[MAXN];
ll qpow(ll d,ll c){//快速幂
    ll res = 1;
    while(c){
        if(c&1) res=res*d%med;
        d=d*d%med;c>>=1;
    }return res;
}
inline ll niyuan(ll x){return qpow(x,med-2);}
void initjc(){//初始化阶乘
    jc[0] = 1;
    rep(i,1,MAXN-1) jc[i] = jc[i-1]*i%med;
}
inline int C(int n,int m){//n是下面的
    if(n<m) return 0;
    return jc[n]*niyuan(jc[n-m])%med*niyuan(jc[m])%med;
}
int main(){
    initjc();
    int n,m;
    while(cin>>n>>m) cout<<C(n,m)<<endl;
}
\end{lstlisting}

\subsection{矩阵快速幂}
\begin{lstlisting}
struct Matrix{
	ll a[MAXN][MAXN];
	
	Matrix(ll x=0){//感觉是特别好的初始化,从hjt那里学(抄)来的 
		for(int i=0;i<n;i++){
			for(int j=0;j<n;j++){
				a[i][j]=x*(i==j);//这句特简洁		
			}
		}
	}
	
	Matrix operator *(const Matrix &b)const{//通过重载运算符实现矩阵乘法 
		Matrix res(0);
		for(int i=0;i<n;i++){
			for(int j=0;j<n;j++){
				for(int k=0;k<n;k++){
					ll &ma = res.a[i][j];
					ma = (ma+a[i][k]*b.a[k][j])%mod;
				}
			}
		}
		return res;
	}
};

Matrix qpow(Matrix d,ll m){//底数和幂次数 
	Matrix res(1);//构造E单位矩阵 
	while(m){
		if(m&1){
			m--;//其实这句是可以不要的 
			res=res*d;
		}
		d=d*d;
		m>>=1;
	}
	return res; 
}
\end{lstlisting}

\subsection{高斯消元}
\begin{lstlisting}
待补充
\end{lstlisting}

\subsection{三点求圆心}
\begin{lstlisting}
struct point{
	double x;
	double y;
};

point cal(point a,point b,point c){
	double x1 = a.x;double y1 = a.y;
	double x2 = b.x;double y2 = b.y;
	double x3 = c.x; double y3 = c.y;
	double a1 = 2*(x2-x1); double a2 = 2*(x3-x2);
	double b1 = 2*(y2-y1); double b2 = 2*(y3-y2);
	double c1 = x2*x2 + y2*y2 - x1*x1 - y1*y1;
	double c2 = x3*x3 + y3*y3 - x2*x2 - y2*y2;
	double rx = (c1*b2-c2*b1)/(a1*b2-a2*b1);
	double ry = (c2*a1-c1*a2)/(a1*b2-a2*b1);
	return point{rx,ry};
}
\end{lstlisting}

\subsection{欧拉降幂}
$$
a^b \equiv \begin{cases}
a^{b\%\phi(p)} , & \gcd(a,p)=1\\
a^b , & \gcd(a,p)\neq 1,b<\phi(p)\\
a^{b\%\phi(p)+\phi(p)} , & \gcd(a,p)\neq 1 , b\geq \phi(p)\\
\end{cases}
(\mod p)
$$

\section{字符串}

\subsection{FFT解决字符串匹配问题}
可以用来解决含有通配符的字符串匹配问题
定义匹配函数 $$(x,y) = (A_x-B_x)^2$$
如果两个字符相同,则满足 $C(x,y)=0$\\
定义模式串和文本串x位置对齐时候的完全匹配函数为
$$P(x)=\sum C(i,x+i)$$
模式串在位置x上匹配时,$p(x)=0$\\
通过将模式串reverse后卷积,可以快速处理每个位置x上的完全匹配函数$P(x)$
同理,如果包含通配符,则设通配符的值为0,可以构造损失函数
$$C(x,y)=(A_x-B_x)^2 \cdot A_x \cdot B_x=A_x^3 B_x+A_xB_x^3-2A_x^2B_x^2$$
通过三次FFT即可求得每个位置上的P(x)

\subsection{后缀数组SA+LCP}
LCP(i,j) 后缀i和后缀j的最长公共前缀
\begin{lstlisting}
int n,m;
string s;
int rk[MAXN],sa[MAXN],c[MAXN],rk2[MAXN];
//sa[i]存排名i的原始编号 rk[i]存编号i的排名 第二关键字rk2
inline void get_SA(){
    rep(i,1,n) ++c[rk[i]=s[i]];//基数排序
    rep(i,2,m) c[i] += c[i-1];
    //c做前缀和,可以知道每个关键字的排名最低在哪里
    repb(i,n,1) sa[c[rk[i]]--] = i;//记录每个排名的原编号

    for(int w=1;w<=n;w<<=1){//倍增
        int num = 0;
        rep(i,n-w+1,n) rk2[++num] = i;//没有第二关键字的排在前面
        rep(i,1,n) if(sa[i]>w) rk2[++num] = sa[i]-w;
        //编号sa[i]大于w的才能作为编号sa[i]-w的第二关键字
        rep(i,1,m) c[i] = 0;
        rep(i,1,n) ++c[rk[i]];
        rep(i,2,m) c[i]+=c[i-1];
        repb(i,n,1) sa[c[rk[rk2[i]]]--]=rk2[i],rk2[i]=0;
        //同一个桶中按照第二关键字排序
        swap(rk,rk2);
        //这时候的rk2时这次排序用到的上一轮的rk,要计算出新的rk给下一轮排序

        rk[sa[1]]=1,num=1;
        rep(i,2,n)
            rk[sa[i]] = (rk2[sa[i]]==rk2[sa[i-1]]&&rk2[sa[i]+w]==rk2[sa[i-1]+w])?num:++num;
        //下一次排名的第一关键字,相同的两个元素排名也相同
        if(num==n) break;//rk都唯一时,排序结束
        m=num;
    }
}
int height[MAXN];
inline void get_height(){
    int k = 0,j;
    rep(i,1,n) rk[sa[i]] = i;
    rep(i,1,n){
        if(rk[i]==1) continue;//第一名往前没有前缀
        if(k) k--;//h[i]>=h[i-1]-1 即height[rk[i]]>=height[rk[i-1]]-1
        j = sa[rk[i]-1];//找排在rk[i]前面的
        while(j+k<=n&&i+k<=n&&s[i+k]==s[j+k]) ++k;//逐字符比较
        //因为每次k只会-1,故++k最多只会加2n次
        height[rk[i]] = k;
    }
}
inline void solve(){
    cin>>s;
    s = ' '+s;
    n = s.size()-1,m = 122;//m为字符个数'z'=122
    get_SA();
    rep(i,1,n) cout<<sa[i]<<' ';
    cout<<endl;
}
\end{lstlisting}

\section{STL等小技巧}

\subsection{集合set}
还可以通过lower\_bound和upper\_bound返回迭代器来找前驱,后继
\begin{lstlisting}
//并交集
vector<int> ANS;
set_union(s1.begin(),s1.end(),s2.begin(),s2.end(),inserter(ANS,ANS.begin()));//set_intersection()

//通过迭代器遍历集合
set<char>::iterator iter = temp1.begin();
while (iter!=temp1.end()){
	cout<<*iter;
	iter++;
}
\end{lstlisting}

\subsection{快读快写(短)}
\begin{lstlisting}
template<class T>inline void read(T &x){x=0;char o,f=1;while(o=getchar(),o<48)if(o==45)f=-f;do x=(x<<3)+(x<<1)+(o^48);while(o=getchar(),o>47);x*=f;}
template<class T>
void wt(T x){//快写
   if(x < 0) putchar('-'), x = -x;
   if(x >= 10) wt(x / 10);
   putchar('0' + x % 10);
}
\end{lstlisting}

\subsection{GCD(压行)}
\begin{lstlisting}
ll gcd(ll a,ll b){ while(b^=a^=b^=a%=b); return a; }
\end{lstlisting}

\subsection{计时}
\begin{lstlisting}
inline double run_time(){
    return 1.0*clock()/CLOCKS_PER_SEC;
}
\end{lstlisting}
%==============================正文部分==============================%
\end{document}