%==============================常用宏包、环境==============================%
\documentclass[twocolumn,a4]{article}
\usepackage{xeCJK} % For Chinese characters
\usepackage{amsmath, amsthm}
\usepackage{listings,xcolor}
\usepackage{geometry} % 设置页边距
\usepackage{fontspec}
\usepackage{graphicx}
\usepackage{float} %设置图片浮动位置的宏包
\usepackage{subfigure} %插入多图时用子图显示的宏包
\usepackage{fancyhdr} % 自定义页眉页脚
\setsansfont{Consolas} % 设置英文字体
\setmonofont[Mapping={}]{Consolas} % 英文引号之类的正常显示,相当于设置英文字体
\geometry{left=1cm,right=1cm,top=2cm,bottom=0.5cm} % 页边距
\setlength{\columnsep}{30pt}
% \setlength\columnseprule{0.4pt} % 分割线
%==============================常用宏包、环境==============================%

%==============================页眉、页脚、代码格式设置==============================%
% 页眉、页脚设置
\pagestyle{fancy}
% \lhead{CUMTB}
\lhead{\CJKfamily{hei} 泡泡猿专用模板}
\rhead{第 \thepage 页}
% \rhead{Page \thepage}
\chead{\leftmark} 
\lfoot{} 
\cfoot{}
\rfoot{}
\renewcommand{\headrulewidth}{0.4pt} 
\renewcommand{\footrulewidth}{0.4pt}

% 代码格式设置
\lstset{
    language    = c++,
    numbers     = left,
    numberstyle = \tiny,
    breaklines  = true,
    captionpos  = b,
    tabsize     = 4,
    frame       = shadowbox,
    columns     = fullflexible,
    commentstyle = \color[RGB]{0,128,0},
    keywordstyle = \color[RGB]{0,0,255},
    basicstyle   = \small\ttfamily,
    stringstyle  = \color[RGB]{148,0,209}\ttfamily,
    rulesepcolor = \color{red!20!green!20!blue!20},
    showstringspaces = false,
}
%==============================页眉、页脚、代码格式设置==============================%

%==============================标题和目录==============================%
\title{\CJKfamily{hei} \bfseries 泡泡猿ACM模板}
\author{Rand0w \& REXWIND \& Dallby}
\renewcommand{\today}{\number\year 年 \number\month 月 \number\day 日}

\begin{document}\small
\begin{titlepage}
\maketitle
\begin{figure}[H] %H为当前位置,!htb为忽略美学标准,htbp为浮动图形
\centering %图片居中
\includegraphics[width=0.7\textwidth]{1.jpg} %插入图片,[]中设置图片大小,{}中是图片文件名
\end{figure}
\end{titlepage}

\newpage
\pagestyle{empty}
\renewcommand{\contentsname}{目录}
\tableofcontents
\newpage\clearpage
\newpage
\pagestyle{fancy}
\setcounter{page}{1}   %new page
%==============================标题和目录==============================%

%==============================正文部分==============================%
\section{头文件}
\subsection{头文件(Rand0w)}
\begin{lstlisting}
#include <bits/stdc++.h>
//#include <bits/extc++.h>
//using namespace __gnu_pbds;
//using namespace __gnu_cxx;
using namespace std;
#pragma optimize(2)
//#pragma GCC optimize("Ofast,no-stack-protector")
//#pragma GCC target("sse,sse2,sse3,ssse3,sse4,popcnt,abm,mmx,avx,avx2,tune=native")
#define rbset(T) tree<T,null_type,less<T>,rb_tree_tag,tree_order_statistics_node_update>
const int inf = 0x7FFFFFFF;
typedef long long ll;
typedef double db;
typedef long double ld;
template<class T>inline void MAX(T &x,T y){if(y>x)x=y;}
template<class T>inline void MIN(T &x,T y){if(y<x)x=y;}
namespace FastIO
{
char buf[1 << 21], buf2[1 << 21], a[20], *p1 = buf, *p2 = buf, hh = '\n';
int p, p3 = -1;
void read() {}
void print() {}
inline int getc()
{
return p1 == p2 && (p2 = (p1 = buf) + fread(buf, 1, 1 << 21, stdin), p1 == p2) ? EOF : *p1++;
}
inline void flush()
{
fwrite(buf2, 1, p3 + 1, stdout), p3 = -1;
}
template <typename T, typename... T2>
inline void read(T &x, T2 &... oth)
{
int f = 0;x = 0;char ch = getc();
while (!isdigit(ch)){if (ch == '-')f = 1;ch = getc();}
while (isdigit(ch)){x = x * 10 + ch - 48;ch = getc();}
x = f ? -x : x;read(oth...);
}
template <typename T, typename... T2>
inline void print(T x, T2... oth)
{
if (p3 > 1 << 20)flush();
if (x < 0)buf2[++p3] = 45, x = -x;
do{a[++p] = x % 10 + 48;}while (x /= 10);
do{buf2[++p3] = a[p];}while (--p);
buf2[++p3] = hh;
print(oth...);
}
} // namespace FastIO
#define read FastIO::read
#define print FastIO::print
#define flush FastIO::flush
#define spt fixed<<setprecision
#define endll '\n'
#define mul(a,b,mod) (__int128)(a)*(b)%(mod) 
#define pii(a,b) pair<a,b>
#define pow powmod
#define X first
#define Y second
#define lowbit(x) (x&-x)
#define MP make_pair
#define pb push_back
#define pt putchar
#define yx_queue priority_queue
#define lson(pos) (pos<<1)
#define rson(pos) (pos<<1|1)
#define y1 code_by_Rand0w
#define yn A_muban_for_ACM
#define j1 it_is just_an_eastegg
#define lr hope_you_will_be_happy_to_see_this
#define int long long
#define rep(i, a, n) for (register int i = a; i <= n; ++i)
#define per(i, a, n) for (register int i = n; i >= a; --i)
const ll llinf = 4223372036854775851;
const ll mod = (0 ? 1000000007 : 998244353);
ll pow(ll a,ll b,ll md=mod) {ll res=1;a%=md; assert(b>=0); for(;b;b>>=1){if(b&1)res=mul(res,a,md);a=mul(a,a,md);}return res;}
const ll mod2 = 999998639;
const int m1 = 998244353;
const int m2 = 1000001011;
const int pr=233;
const double eps = 1e-7;
const int maxm= 1;
const int maxn = 510000;
void work()
{
	
}
signed main()
{
   #ifndef ONLINE_JUDGE
   //freopen("in.txt","r",stdin);
	//freopen("out.txt","w",stdout);
#endif
	//std::ios::sync_with_stdio(false);
	//cin.tie(NULL);
	int t = 1;
	//cin>>t;
	for(int i=1;i<=t;i++){
		//cout<<"Case #"<<i<<":"<<endll;
		work();
	}
	return 0;
}
\end{lstlisting}
\newpage
\subsection{头文件(REXWind)}
\begin{lstlisting}
#include<iostream>
#include<cstring>
#include<cstdio>
#include<algorithm>
#include<vector>
#include<map>
#include<queue>
#include<cmath>
using namespace std;

template<class T>inline void read(T &x){x=0;char o,f=1;while(o=getchar(),o<48)if(o==45)f=-f;do x=(x<<3)+(x<<1)+(o^48);while(o=getchar(),o>47);x*=f;}
int cansel_sync=(ios::sync_with_stdio(0),cin.tie(0),0);
#define ll long long
#define ull unsigned long long
#define rep(i,a,b) for(int i=(a);i<=(b);i++)
#define repb(i,a,b) for(int i=(a);i>=b;i--)
#define mkp make_pair
#define ft first
#define sd second
#define log(x) (31-__builtin_clz(x))
#define INF 0x3f3f3f3f
typedef pair<int,int> pii;
typedef pair<ll,ll> pll;
ll gcd(ll a,ll b){ while(b^=a^=b^=a%=b); return a; }
//#define INF 0x7fffffff

void solve(){
	
}

int main(){
	int z;
	cin>>z;
	while(z--) solve();
}
\end{lstlisting}
\subsection{头文件(Dallby)}
\begin{lstlisting}
#include<bits/stdc++.h>
cout<<"hello<<endl;
\end{lstlisting}

\section{数据结构}

\subsection{扫描线}
扫描线是离散化后,使用类似权值线段树来维护每个截面上的线段长度。\\
通过把二维平面上的四边形拆分成入边和出边两段,在遇到边的时候对对应的区间进行区间加/减即可。\\
每个节点上需要维护被完全覆盖的次数和实际长度。
\begin{lstlisting}
#define ls (x<<1)
#define rs (x<<1|1)//这种方法感觉还挺好的

int cansel_sync=(ios::sync_with_stdio(0),cin.tie(0),0);
const int MAXN = 2e5+5;//这里要开n的两倍
//线结构体
struct Line{
    ll l,r,h;
    int qz;//记录位置和权值
    bool operator < (Line &rhs){
        return h < rhs.h;
    }
}line[MAXN];
int n;
ll x1,y1,x2,y2;
ll X[MAXN];
//线段树
struct Segt{
    int l,r;//是X的下标,即离散化后的
    int sum;//sum是被完全覆盖的次数
    ll len;//len是区间内被盖住的长度
    //因为每次查询都是查询根节点,所以这边不需要懒惰标记
}t[MAXN<<3];//一个边有两个点,所以这里要开8倍
void build(int x,int l,int r){
    t[x].l = l;t[x].r = r;
    t[x].len = t[x].sum = 0;
    if(l==r) return;//到了叶子节点
    int mid = (l+r)>>1;
    build(ls,l,mid);
    build(rs,mid+1,r);
}
void push_up(int x){
    int l = t[x].l,r = t[x].r;
    if(t[x].sum) t[x].len = X[r+1]-X[l];//x的区间是X[l]到X[r+1]-1
    else t[x].len = t[ls].len + t[rs].len;//合并儿子的信息
}
void update(int x,int L,int R,int v){//这里的LR存的是实际值
    //这里如果是线段L,R,线段树上是L到R-1的部分维护
    int l = t[x].l,r = t[x].r;
    if(X[r+1]<=L||R<=X[l]) return;//加等于,不然会搞到无辜的线
    if(L<=X[l]&&X[r+1]<=R){
        t[x].sum += v;//修改覆盖次数
        push_up(x);
        return;
    }
    update(ls,L,R,v);
    update(rs,L,R,v);
    push_up(x);
}
int main(){
    cin>>n;
    rep(i,1,n){
        cin>>x1>>y1>>x2>>y2;
        X[2*i-1] = x1,X[2*i] = x2;//一会儿离散化要用的,这里存实际值
        line[2*i-1] = Line{x1,x2,y1,1};//开始的线
        line[2*i] = Line{x1,x2,y2,-1};//结束的线
    }
    n<<=1;//line的数量是四边形数量的2倍
    sort(line+1,line+1+n);
    sort(X+1,X+1+n);
    int tot = unique(X+1,X+n+1)-(X+1);//去除重复相邻元素,并且tot记录总数
    build(1,1,tot-1);//为什么是tot-1?
    //因为线段树只需要维护X[1]到X[tot]-1这一段的,实际长度是向右贴的
    ll res = 0;
    rep(i,1,n-1){//每次高度是line[i+1].h-line[i].h,所以是到n-1就行
        update(1,line[i].l,line[i].r,line[i].qz);//扫描线加入线段树
        res += t[1].len*(line[i+1].h-line[i].h);
    }
    cout<<res<<endl;
}
\end{lstlisting}

\section{数论}

\subsection{欧拉筛}
$O(n)$筛素数
\begin{lstlisting}
int primes[maxn+5],tail;
bool is_prime[maxn+5];
void euler(){
   is_prime[1] = 1;
   for (int i = 2; i < maxn; i++)
   {
      if (!is_prime[i])
      primes[++tail]=i;
      for (int j = 0; j < primes.size() && i * primes[j] < maxn; j++)
      {
         is_prime[i * primes[j]] = 1;
         if ((i % primes[j]) == 0)
            break;
      }
   }
}
\end{lstlisting}

\subsection{Exgcd}
求出$ax+by=gcd(a,b)$的一组可行解 $O(logn)$ 
\begin{lstlisting}
void Exgcd(ll a,ll b,ll &d,ll &x,ll &y){
	if(!b){d=a;x=1;y=0;}
	else{Exgcd(b,a%b,d,y,x);y-=x*(a/b);}
}
\end{lstlisting}

\subsection{Excrt 扩展中国剩余定理}
求解同余方程组
$\begin{cases}
	\begin{aligned}
	x \ \% \ b_1  &\equiv \  a_1\\
	x \ \% \ b_2  &\equiv \ a_2\\
	           		& \ \vdots   \\
	x \ \% \ b_n  &\equiv  \ a_n
	\end{aligned}
\end{cases}$
\begin{lstlisting}
int excrt(int a[],int b[],int n){
    int lc=1;
    for(int i=1;i<=n;i++)
        lc=lcm(lc,a[i]);
    for(int i=1;i<n;i++){
        int p,q,g;
        g=exgcd(a[i],a[i+1],p,q);
        int k=(b[i+1]-b[i])/g;
        q=-q;p*=k;q*=k;
        b[i+1]=a[i]*p%lc+b[i];
        b[i+1]%=lc;
        a[i+1]=lcm(a[i],a[i+1]);
    }
    return (b[n]%lc+lc)%lc;
}
\end{lstlisting}

\subsection{线性求逆元}
\begin{lstlisting}
void init(int p){
	inv[1] = 1;
	for(int i=2;i<=n;i++){
		inv[i] = (ll)(p-p/i)*inv[p%i]%p;
	}
}
\end{lstlisting}

\subsection{组合数}
预处理阶乘,并通过逆元实现相除
\begin{lstlisting}
ll jc[MAXN];
ll qpow(ll d,ll c){//快速幂
    ll res = 1;
    while(c){
        if(c&1) res=res*d%med;
        d=d*d%med;c>>=1;
    }return res;
}
inline ll niyuan(ll x){return qpow(x,med-2);}
void initjc(){//初始化阶乘
    jc[0] = 1;
    rep(i,1,MAXN-1) jc[i] = jc[i-1]*i%med;
}
inline int C(int n,int m){//n是下面的
    if(n<m) return 0;
    return jc[n]*niyuan(jc[n-m])%med*niyuan(jc[m])%med;
}
int main(){
    initjc();
    int n,m;
    while(cin>>n>>m) cout<<C(n,m)<<endl;
}
\end{lstlisting}

\subsection{矩阵快速幂}
\begin{lstlisting}
struct Matrix{
	ll a[MAXN][MAXN];
	
	Matrix(ll x=0){//感觉是特别好的初始化,从hjt那里学(抄)来的 
		for(int i=0;i<n;i++){
			for(int j=0;j<n;j++){
				a[i][j]=x*(i==j);//这句特简洁		
			}
		}
	}
	
	Matrix operator *(const Matrix &b)const{//通过重载运算符实现矩阵乘法 
		Matrix res(0);
		for(int i=0;i<n;i++){
			for(int j=0;j<n;j++){
				for(int k=0;k<n;k++){
					ll &ma = res.a[i][j];
					ma = (ma+a[i][k]*b.a[k][j])%mod;
				}
			}
		}
		return res;
	}
};

Matrix qpow(Matrix d,ll m){//底数和幂次数 
	Matrix res(1);//构造E单位矩阵 
	while(m){
		if(m&1){
			m--;//其实这句是可以不要的 
			res=res*d;
		}
		d=d*d;
		m>>=1;
	}
	return res; 
}
\end{lstlisting}

\subsection{高斯消元}
\begin{lstlisting}
待补充
\end{lstlisting}

\subsection{欧拉降幂}
$$
a^b \equiv \begin{cases}
a^{b\%\phi(p)} , & \gcd(a,p)=1\\
a^b , & \gcd(a,p)\neq 1,b<\phi(p)\\
a^{b\%\phi(p)+\phi(p)} , & \gcd(a,p)\neq 1 , b\geq \phi(p)\\
\end{cases}
(\mod p)
$$

\section{计算几何}

\subsection{三点求圆心}
\begin{lstlisting}
struct point{
	double x;
	double y;
};

point cal(point a,point b,point c){
	double x1 = a.x;double y1 = a.y;
	double x2 = b.x;double y2 = b.y;
	double x3 = c.x; double y3 = c.y;
	double a1 = 2*(x2-x1); double a2 = 2*(x3-x2);
	double b1 = 2*(y2-y1); double b2 = 2*(y3-y2);
	double c1 = x2*x2 + y2*y2 - x1*x1 - y1*y1;
	double c2 = x3*x3 + y3*y3 - x2*x2 - y2*y2;
	double rx = (c1*b2-c2*b1)/(a1*b2-a2*b1);
	double ry = (c2*a1-c1*a2)/(a1*b2-a2*b1);
	return point{rx,ry};
}
\end{lstlisting}

\section{字符串}

\subsection{KMP}
\begin{lstlisting}
const int MAXN = 2e6+5;
int pi[MAXN];//MAXN记得开大一点,因为这里要存到m+n+1长度的 
vector<int> res;//储存答案
 
void getpi(const string &s){ //求s的前缀函数
	pi[0]=0;
	int j=0;
	rep(i,1,s.length()-1){
		while(j>0&&s[i]!=s[j]) j=pi[j-1];//找到合适且最长的j 
		if(s[i]==s[j])j++;//能成功匹配的情况 
		pi[i]=j;
	}
}

void kmp(string s,string t){ //在主串t中找模式串s 
	getpi(s+'#'+t);
	int n=(int)s.length(),m=(int)t.length();
	rep(i,n+1,m+n+1-1)
		if(pi[i]==n) res.push_back(i-2*s.size()); //i-2n计算得左端点 
}
\end{lstlisting}

\subsection{AC自动机}
\begin{lstlisting}
const int MAXN = 1e5+5;
int jdbh[MAXN];//记录第i个模式串对应的节点编号
int cntcx[MAXN];//记录第i个模式串出现的次数
inline int idx(char c){return c-'a';}
struct Node{
    int son[26],flag,fail;//cnt记录次数,flag记录编号
    void clr(){
        memset(son,0,sizeof(son));
        flag=0;
    }
}trie[MAXN*10];
int n,cntt;//cntt记录总点数
string s,ms[166];
int maxx;
queue<int>q;
inline void insert(string &s,int num){
    int siz = s.size(),v,u=1;
    rep(i,0,siz-1){
        v = idx(s[i]);
        if(!trie[u].son[v]){trie[u].son[v] = ++cntt;trie[cntt].clr();}
        u = trie[u].son[v];
    }
    trie[u].flag = num;//标记为单词,flag记录编号
    //保证每个模式串只出现一次
    cntcx[num] = 0;
    jdbh[num] = u;//记录当前单词对应的节点编号
}
inline void getfail(){
    rep(i,0,25) trie[0].son[i] = 1;
    trie[0].flag = 0;
    q.push(1);
    trie[1].fail = 0;
    int u,v,ufail;
    while(!q.empty()){
        u = q.front();q.pop();
        rep(i,0,25){
            v = trie[u].son[i];
            ufail = trie[u].fail;
            if(!v){trie[u].son[i]=trie[ufail].son[i];continue;}//画好一条跳fail的路
            trie[v].fail = trie[ufail].son[i];
            q.push(v);
        }
    }
}
inline void query(string &s){
    int siz = s.size(),u = 1,v,k;
    rep(i,0,siz-1){
        v = idx(s[i]);
        k = trie[u].son[v];
        while(k){
            if(trie[k].flag){
                cntcx[trie[k].flag]++;//计数
                maxx = max(maxx,cntcx[trie[k].flag]);
            }
            k = trie[k].fail;//跳fail
        }
        u = trie[u].son[v];//这一句其实也有跳fail的功能,很精妙
    }
}
inline void solve(){
    cntt = 1;
    trie[0].clr();
    trie[1].clr();
    rep(i,1,n){
        cin>>ms[i];
        insert(ms[i],i);
    }
    getfail();
    cin>>s;
    maxx = 0;
    query(s);
    cout<<maxx<<endl;
    rep(i,1,n){
        if(cntcx[i]==maxx) cout<<ms[i]<<endl;
    }
}
\end{lstlisting}

\subsection{FFT解决字符串匹配问题}
可以用来解决含有通配符的字符串匹配问题
定义匹配函数 $$(x,y) = (A_x-B_x)^2$$
如果两个字符相同,则满足 $C(x,y)=0$\\
定义模式串和文本串x位置对齐时候的完全匹配函数为
$$P(x)=\sum C(i,x+i)$$
模式串在位置x上匹配时,$p(x)=0$\\
通过将模式串reverse后卷积,可以快速处理每个位置x上的完全匹配函数$P(x)$
同理,如果包含通配符,则设通配符的值为0,可以构造损失函数
$$C(x,y)=(A_x-B_x)^2 \cdot A_x \cdot B_x=A_x^3 B_x+A_xB_x^3-2A_x^2B_x^2$$
通过三次FFT即可求得每个位置上的P(x)\\
以下是用FFT解决普通字符串匹配问题的代码\\
即实现KMP的功能,复杂度较高,为$O(nlog_n)$\\
\begin{lstlisting}
void solve(){
    limit = 1,l=0;
    cin>>n>>m;
    cin>>s1>>s2;
    rep(i,0,n-1) B[i].x = s1[i]-'a'+1;
    rep(i,0,m-1) A[i].x = s2[i]-'a'+1;
    double T = 0;
    //T = sigma A[i]^A[i] i=0~m-1
    rep(i,0,m-1) T += A[i].x*A[i].x;
    //f[x] = sigma B[i]^B[i] i=0~x
    f[0] = B[0].x*B[0].x;
    rep(i,1,n-1) f[i] = f[i-1]+B[i].x*B[i].x;
    //g[x] = S[i]*B[j] i+j==x
    reverse(A,A+m);//S = A.reverse
    //FFT预处理
    while(limit<=n+m-2) limit<<=1,l++;
    rep(i,0,limit-1)
        r[i]= ( r[i>>1]>>1 )| ( (i&1)<<(l-1) );
    
    FFT(A,1);FFT(B,1);
    rep(i,0,limit) A[i]=A[i]*B[i];
    FFT(A,-1);
    rep(i,0,n-1) g[i] = (int)(A[i].x/limit+0.5);//四舍五入
    
    //T + f(x) - f(x-m) - 2g(x);
    double tmp;
    rep(x,m-1,n-1){
        tmp = T+f[x]-2*g[x];
        if(x!=m-1) tmp -= f[x-m];
        //cout<<tmp<<' ';
        if(fabs(tmp)<eps) cout<<x-(m-1)+1<<endl;//输出匹配上的位置
    }
    cout<<endl;
}
\end{lstlisting}

\subsection{字符串哈希}
快速取子串哈希值
\begin{lstlisting}
const int b = 131;//推荐的base,可以选其他质数
void init(int n){//初始化 
    pw[0] = 1;
    for (int i = 1; i <= n; i ++ ) {
        h[i] = h[i-1]*b + str[i];//做每个前缀的哈希值 
        pw[i] = pw[i-1]*b;//预处理b^k的值 
    }
}
// 计算子串 str[l ~ r] 的哈希值
ull get(int l, int r) {
    return h[r] - h[l-1]*pw[r-l+1];
}
\end{lstlisting}


\subsection{后缀数组SA+LCP}
LCP(i,j) 后缀i和后缀j的最长公共前缀
\begin{lstlisting}
int n,m;
string s;
int rk[MAXN],sa[MAXN],c[MAXN],rk2[MAXN];
//sa[i]存排名i的原始编号 rk[i]存编号i的排名 第二关键字rk2
inline void get_SA(){
    rep(i,1,n) ++c[rk[i]=s[i]];//基数排序
    rep(i,2,m) c[i] += c[i-1];
    //c做前缀和,可以知道每个关键字的排名最低在哪里
    repb(i,n,1) sa[c[rk[i]]--] = i;//记录每个排名的原编号

    for(int w=1;w<=n;w<<=1){//倍增
        int num = 0;
        rep(i,n-w+1,n) rk2[++num] = i;//没有第二关键字的排在前面
        rep(i,1,n) if(sa[i]>w) rk2[++num] = sa[i]-w;
        //编号sa[i]大于w的才能作为编号sa[i]-w的第二关键字
        rep(i,1,m) c[i] = 0;
        rep(i,1,n) ++c[rk[i]];
        rep(i,2,m) c[i]+=c[i-1];
        repb(i,n,1) sa[c[rk[rk2[i]]]--]=rk2[i],rk2[i]=0;
        //同一个桶中按照第二关键字排序
        swap(rk,rk2);
        //这时候的rk2时这次排序用到的上一轮的rk,要计算出新的rk给下一轮排序

        rk[sa[1]]=1,num=1;
        rep(i,2,n)
            rk[sa[i]] = (rk2[sa[i]]==rk2[sa[i-1]]&&rk2[sa[i]+w]==rk2[sa[i-1]+w])?num:++num;
        //下一次排名的第一关键字,相同的两个元素排名也相同
        if(num==n) break;//rk都唯一时,排序结束
        m=num;
    }
}
int height[MAXN];
inline void get_height(){
    int k = 0,j;
    rep(i,1,n) rk[sa[i]] = i;
    rep(i,1,n){
        if(rk[i]==1) continue;//第一名往前没有前缀
        if(k) k--;//h[i]>=h[i-1]-1 即height[rk[i]]>=height[rk[i-1]]-1
        j = sa[rk[i]-1];//找排在rk[i]前面的
        while(j+k<=n&&i+k<=n&&s[i+k]==s[j+k]) ++k;//逐字符比较
        //因为每次k只会-1,故++k最多只会加2n次
        height[rk[i]] = k;
    }
}
inline void solve(){
    cin>>s;
    s = ' '+s;
    n = s.size()-1,m = 122;//m为字符个数'z'=122
    get_SA();
    rep(i,1,n) cout<<sa[i]<<' ';
    cout<<endl;
}
\end{lstlisting}

\subsection{后缀自动机SAM}
\begin{lstlisting}
struct state{
    int len,link;
    map<char,int> nxt;//也可以用数组,空间换时间
};
state sta[MAXN<<1];//状态数需要设定为两倍
int sz,last;//sz为自动机大小
inline void init_SAM(){
    sta[0].len = 0;sta[0].link = -1;//虚拟状态t0
    sz = 1;
    last = 0;
}
int cnt[MAXN<<1];
void SAM_extend(char c){
    int cur = sz++;
    cnt[cur] = 1;
    sta[cur].len = sta[last].len+1;
    int p = last;
    //沿着last的link添加到c的转移,直到找到已经有c转移的状态p
    while(p!=-1&&!sta[p].nxt.count(c)){
        sta[p].nxt[c] = cur;
        p = sta[p].link;
    }
    if(p==-1) sta[cur].link = 0;//情况1,没有符合的p
    else{
        int q = sta[p].nxt[c];
        if(sta[q].len==sta[p].len+1)//情况2,稳定的转移(lenq=lenp+1,前面没有增加)
            sta[cur].link = q;
        else{//情况3,把q的lenp+1的部分拿出来(clone),p到clone的转移是稳定的
            int clone = sz++;
            cnt[clone] = 0;
            sta[clone].len = sta[p].len+1;
            sta[clone].nxt = sta[q].nxt;
            sta[clone].link = sta[q].link;
            while(p!=-1 && sta[p].nxt[c]==q){//把向q的转移指向clone
                sta[p].nxt[c]=clone;
                p=sta[p].link;
            }
            sta[q].link = sta[cur].link = clone;//clone是q的后缀,故linkq=clone
        }
    }
    last = cur;//sta[last]包含目前处理的整个前缀!
}
string s;
vector<int> e[MAXN<<1];
void dfs(int now){
    for(auto to:e[now]){
        dfs(to);
        cnt[now] += cnt[to];
    }
}
inline void solve(){
    cin>>s;
    init_SAM();
    int siz = s.size();
    rep(i,0,siz-1) SAM_extend(s[i]);
    rep(i,1,sz-1) e[sta[i].link].push_back(i);//link边反过来构造树
    dfs(0);
    ll maxx = 0;
    rep(i,1,sz-1)
        if(cnt[i]!=1) maxx = max(maxx,1ll*cnt[i]*sta[i].len);
    cout<<maxx<<endl;
}
int main(){
    solve();    
}
//P3804 【模板】后缀自动机 (SAM)
//https://www.luogu.com.cn/problem/P3804
\end{lstlisting}

\section{其他}

\subsection{ST表求RMQ}
$O(nlog_n)$预处理,$O(1)$查询
\begin{lstlisting}
#define log(x) (31-__builtin_clz(x))//谢谢hjt
const int MAXN = 1e5+10;
const int LOGN = log(MAXN)/log(2)+5;//这里要开大一点,之前因为没开大翻车了 

int M[MAXN][LOGN]; 
int a[MAXN];
int z,m,n;

void init(){//初始化,复杂度O(nlogn) 
	for(int i=1;i<=n;i++) M[i][0]=i;//长度为1的区间最值是自己 
	for(int j=1;j<=LOGN;j++){
		for(int i=1;i<=n-(1<<j)+1;i++){
			if(a[M[i][j-1]]<a[M[i+(1<<(j-1))][j-1]]) M[i][j] = M[i][j-1];//这里以最小值为例 
			else M[i][j] = M[i+(1<<j-1)][j-1];
		}
	} 
}

int query(int l,int r){
	int k = log(r-l+1)/log(2);//向下取整
	if(a[M[l][k]]<a[M[r-(1<<k)+1][k]]) return M[l][k];
	else return M[r-(1<<k)+1][k];
}
\end{lstlisting}

\subsection{莫队}
\begin{lstlisting}
int cnt[MAXN];//记录数字在区间[l,r]内出现的次数
int pos[MAXN],a[MAXN];
ll ans[MAXN];
int n,m,k,res;
struct Q{
    int l,r,k;//k记录原来的编号
    friend bool operator < (Q x,Q y){//同一个分块内r小的排前面;不同分块则按分块靠前的
        return pos[x.l]==pos[y.l]?x.r<y.r:pos[x.l]<pos[y.l];
        //return (pos[a.l]^pos[b.l])?pos[a.l]<pos[b.l]:((pos[a.l]&1)?a.r<b.r:a.r>b.r);
        //这条第一个和==是一样的,后面的是对于左端点在同一奇数块的区间,右端点按升序排列,反之降序
    }
}q[MAXN];

void Add(int pos){
    res -= cnt[a[pos]]*cnt[a[pos]];
    cnt[a[pos]]++;
    res += cnt[a[pos]]*cnt[a[pos]];
}
void Sub(int pos){
    res -= cnt[a[pos]]*cnt[a[pos]];
    cnt[a[pos]]--;
    res += cnt[a[pos]]*cnt[a[pos]];
}
int main(){
    cin>>n>>m>>k;//k为数字范围
    memset(cnt,0,sizeof(cnt));
    int siz = sqrt(n);//每个分块的大小
    rep(i,1,n){
        cin>>a[i];
        pos[i] = i/siz;//分块
    }
    rep(i,1,m){
        cin>>q[i].l>>q[i].r;
        q[i].k = i;//记录原来的编号,用于打乱顺序后的还原
    }
    sort(q+1,q+1+m);
    res = 0;//初始化res
    int l = 1,r = 0;//当前知道的区间
    //因为是闭区间,如果是[1,1]的话则一开始就包含一个元素了
    rep(i,1,m){//莫队的核心,注意加减的顺序
        while(q[i].l<l) Add(--l);
        while(q[i].l>l) Sub(l++);
        while(q[i].r<r) Sub(r--);
        while(q[i].r>r) Add(++r);
        ans[q[i].k] = res;
    }
    rep(i,1,m) cout<<ans[i]<<endl;
}	
\end{lstlisting}

\subsection{带修莫队}
\begin{lstlisting}
int a[MAXN],b[MAXN];//a读入一开始的序列,b记录修改后的
int pos[MAXN];//分块
int cq,cr;//统计查询修改次数
int R[MAXN][3];//0记位置,1记原本的值,2记修改后的值
ll res;
int ans[MAXN];//记录结果
int n,m;
void Add(int x){if(cnt[x]==0)res++;cnt[x]++;}//带修莫队的add和sub有区别
void Sub(int x){if(cnt[x]==1)res--;cnt[x]--;}
struct Q{
    int l,r,k,t;
    friend bool operator < (Q a,Q b){
        return (pos[a.l]^pos[b.l])?pos[a.l]<pos[b.l]:((pos[a.r]^pos[b.r])?a.r<b.r:a.t<b.t);
        //增加第三关键字,询问的先后顺序,用t或者k应该都行
    }
}q[MAXN];
int main(){
    cin>>n>>m;
    cq = cr = 0;
    int siz = pow(n,2.0/3.0);//这么分块最好,别问
    rep(i,1,n){
        cin>>a[i];
        b[i]=a[i];
        pos[i] = i/siz;
    }
    char hc;
    rep(i,1,m){//读入修改和询问
        cin>>hc;
        if(hc=='Q'){
            cin>>q[cq].l>>q[cq].r;
            q[cq].k=cq;q[cq].t=cr;//注意这时候R[cr]还是没有的,这次询问是在R[cr-1]之后的
            cq++;
        }
        else{
            cin>>R[cr][0]>>R[cr][2];
            R[cr][1] = b[R[cr][0]];
            b[R[cr][0]] = R[cr][2];//在b数组中记录更改
            cr++;
        }
    }
    sort(q,q+cq);
    int l=1,r=0,sjc=0;//时间戳
    res = 0;
    rep(i,0,cq-1){
        while(sjc<q[i].t){
            if(l<=R[sjc][0]&&R[sjc][0]<=r)//判断修改是否在该区间内
                Sub(R[sjc][1]),Add(R[sjc][2]);
            a[R[sjc][0]] = R[sjc][2];//在a上也进行更改
            sjc++;
        }
        while(sjc>q[i].t){
            sjc--;
            if(l<=R[sjc][0]&&R[sjc][0]<=r)//判断修改是否在该区间内
                Sub(R[sjc][2]),Add(R[sjc][1]);
            a[R[sjc][0]] = R[sjc][1];//在a上也进行更改
        }
        while(l>q[i].l) Add(a[--l]);
        while(l<q[i].l) Sub(a[l++]);
        while(r<q[i].r) Add(a[++r]);
        while(r>q[i].r) Sub(a[r--]);
        ans[q[i].k] = res;
    }
    rep(i,0,cq-1) cout<<ans[i]<<endl;
}
\end{lstlisting}

\section{STL等小技巧}

\subsection{集合set}
还可以通过lower\_bound和upper\_bound返回迭代器来找前驱,后继
\begin{lstlisting}
//并交集
vector<int> ANS;
set_union(s1.begin(),s1.end(),s2.begin(),s2.end(),inserter(ANS,ANS.begin()));//set_intersection()

//通过迭代器遍历集合
set<char>::iterator iter = temp1.begin();
while (iter!=temp1.end()){
	cout<<*iter;
	iter++;
}
\end{lstlisting}

\subsection{快读快写(短)}
\begin{lstlisting}
template<class T>inline void read(T &x){x=0;char o,f=1;while(o=getchar(),o<48)if(o==45)f=-f;do x=(x<<3)+(x<<1)+(o^48);while(o=getchar(),o>47);x*=f;}
template<class T>
void wt(T x){//快写
   if(x < 0) putchar('-'), x = -x;
   if(x >= 10) wt(x / 10);
   putchar('0' + x % 10);
}
\end{lstlisting}

\subsection{GCD(压行)}
\begin{lstlisting}
ll gcd(ll a,ll b){ while(b^=a^=b^=a%=b); return a; }
\end{lstlisting}

\subsection{计时}
\begin{lstlisting}
inline double run_time(){
    return 1.0*clock()/CLOCKS_PER_SEC;
}
\end{lstlisting}
%==============================正文部分==============================%
\end{document}